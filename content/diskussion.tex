\section{Diskussion}
\label{sec:Diskussion}

\begin{table}
  \centering
  \caption{Abweichung von den Literarturwerten.}
  \label{tab:perc}
  \sisetup{table-format=3.2}
  \begin{tabular}{SSSSSS}
    \toprule
    &&& \multicolumn{3}{c}{Abweichung}\\
    \cmidrule(lr){4-6}
    {$\kappa_\text{Messing}$} & {$\kappa_\text{Aluminium}$} & {$\kappa_\text{Edelstahl}$}
    & {$\text{Messing}$} & {$\text{Aluminium}$} & {$\text{Edelstahl}$}\\
    \si{\watt\per\meter\per\kelvin} & \si{\watt\per\meter\per\kelvin} & \si{\watt\per\meter\per\kelvin}
    & \si{\percent} & \si{\percent} & \si{\percent}\\
    \midrule
    89,04 & 198,47 & 11,33 & 25,80 & 16,26 & 24,47 \\
    \bottomrule
  \end{tabular}
\end{table}

Die Abweichungen von bis zu 25,8\si{\percent} sind dadurch zu erklären, dass das System nicht optimal isoliert ist.
Zudem gibt das Kühlelement zuviel Wärme nach außen und heitzt die Metalle teilweise
mehr auf als das es sie abkühlt. Alles in allem  halten sich die Werte jedoch
relativ gut an den Literaturwerten. Zum statischen Experiment \ref{sec:astat}
ist noch zu erwähnen, dass bereits knapp nach 600\si{\second} die vorgegebenen
maximalen Temperaturen überschritten werden, sodass die Referenzwerte statt
bei 700\si{\second} bei 600\si{\second} genutzt werden müssen.
