\section{Zielführung}
\label{sec:Zielführung}
Das Ziel dieses Versuchs ist es die Ausbreitung von Wärme in einem dünnen
und einem breiten Messingstab, Aluminium und Edelstahl zu untersuchen.




\section{Theorie}
\label{sec:Theorie}
Bei Körpern die keine gleichmäßige Temperatur besitzen kommt es zum Transport
von Wärme von der Stelle mit der höheren Temperatur zu der mit der niedrigeren.
Es gibt drei Möglichkeiten dies zu beschreiben. Die Übertragung durch
durch Wärmestrahlung, Wärmeleitung oder durch Teilchenströmung(Konvektion).

Im Folgenden wird der Wärmetransport per Wärmeleitung untersucht. Dies geschieht
in den genannten Metallen durch frei bewegliche Elektronen und Phononen.

\subsection{Wärmestrom durch Querschnittsfläche}
In dem zu betrachtenden Metall wird die Wärmemenge Q, welche transportiert
wird falls es eine Temperaturdifferenz gibt, gemäß
\begin{equation}
 \symup{d}\,Q = -\kappa A\frac{\partial T}{\partial x}\symup{d}\,t
 \label{eqn:dQ}
\end{equation}
berechnet.
Hierbei ist A die Querschnittsfläche, T die Temperatur und $\kappa$ die
Wärmeleitfähigkeit des Materials. Hieraus lässt sich auch die Wärmestromdichte
\begin{equation}
  j_w = \frac{1}{A}\frac{\symup{d}\,Q}{\symup{d}\,t} = - \kappa
  \frac{\partial T}{\partial x}
  \label{eqn:jw}
\end{equation}
herleiten.

Mit Verwendung der Kontinuitätsgleichung des Wärmestroms lässt sich
dies zu
\begin{equation}
  \frac{\partial T}{\partial t} = \frac{\kappa}{\rho c}\frac{{\partial}^{2} T}
  {\partial {x}^{2}}
  \label{eqn:delT/delt}
\end{equation}
umformen.

\subsection{Temperaturwelle}
Durch periodisches erwärmen und abkühlen entsteht in einem langen Stab
eine Temperaturwelle nach:
\begin{equation}
  T(x,t)= T_{max}\,exp\left(-\sqrt{\frac{\omega\rho c}{2\kappa}}\right)
  \cos\left(\omega t - \sqrt{\frac{\omega\rho c}{2\kappa}}x\right)
  \label{eqn:T(x,t)}
\end{equation}
Da es sich um eine Welle handelt wird die Phasengeschwindigkeit durch
\begin{equation}
  v= \frac{\omega}{k} = \sqrt{\frac{2\kappa\omega}{\rho c}}
  \label{eqn:vpha}
\end{equation}
errechnet.

Die Temperaturwelle ist eine gedämpfte Welle, wobei sich die Dämpfung
durch das Vehältnis der Amplituden $A_{nah}$ zu $A_{fern}$ ergibt. Diese
werden an den Messstellen $x_{nah}$ und $x_{fern}$ ausgelesen.
Zudem werden die Winkelgeschwindigkeit
\begin{equation}
  \omega = \frac{2\pi}{T_{P}}
  \label{eqn:omega}
\end{equation}
und die Phase
\begin{equation}
  \Phi = \frac{2\pi\increment t}{T_P}
  \label{eqn:phi}
\end{equation}
zur Berechnung der Wärmeleitfähigkeit mit $T_P$
als Periodendauer verwendet.
Daraus ergibt sich die Wärmeleitfähigkeit
\begin{equation}
  \kappa = \frac{\rho c (\increment x)^2}{2\increment t \, ln(\frac{A_{nah}}
  {A_{fern}})}
  \label{eqn:kappa}
\end{equation}
mit
\begin{equation}
  \increment x = x_{fern} - x_{nah}
  \label{eqn:deltax}
\end{equation}
als Abstand und $\increment t$ als Phasendifferenz zwischen den Messstellen.



Die Frequenz und Wellenlänge werden durch folgende Formeln bestimmt:
\begin{equation}
  f = \frac{1}{T}
  \label{eqn:Hz}
\end{equation}

\begin{equation}
  \lambda = v\, T = \frac{\increment x}{\increment t} T
  \label{eqn:lambda}
\end{equation}


\subsection{Literaturwerte}

\begin{table}[h]
  \centering
  \label{tab:lit}
  \begin{tabular}{ c c c c }
    \toprule
    $Materialien$ & $ \rho \left[\frac{kg}{m^3}\right]$ &
    c $\left[\frac{J}{kg \, K}\right]$  & $\kappa \left[\frac{W}{m\,K}\right]$
    \\
    \midrule
    $Aluminium$ & 2800 & 830 & 237 \\
    $Messing$ & 8520 & 385 &  120 \\
    $Edelstahl$ & 8000 & 400 & 15 \\
    %$Wasser$ & 0 & 0 & 0 \\
    \bottomrule
  \end{tabular}
  \caption{Literaturwerte \cite{spez} }
\end{table}
\newpage
\subsection{Fehlerrechnung}

Die Mittelwerte werden durch
\begin{equation}
  \overline{x} =\frac{1}{N}\sum_{i=1}^{N} x_i
  \label{eqn:mean}
\end{equation}
bestimmt.

Die Standardabweichung wird durch
\begin{equation}
  \sigma = \sqrt{\left(\frac{1}{N}\sum_{i=1}^{N}(x_i-\overline{x})^2\right)}
  \label{eqn:std}
\end{equation}
 errechnet.
 Zudem ergibt sich die Fehlerfortpflanzung aus
 \begin{equation}
   \increment y = \sqrt{\sum_{i=1}^{n}\left(\frac{\partial y}{\partial x_i}
   \increment x_i\right)^2}
   \label{eqn:gauss}.
\end{equation}
