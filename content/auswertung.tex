\section{Auswertung}
\label{sec:Auswertung}


\subsection{Auswetrung der statischen Messung}
\label{sec:astat}

\begin{figure}
  \centering
  \includegraphics{build/stat.pdf}
  \caption{Die Temperaturverläufe $T1$ und $T4$ und $T5$ und $T8$.}
  \label{fig:plotstat}
\end{figure}

In \ref{fig:plotstat} sind die Temperaturverläufe der
der nicht geheitzten Enden der Metalle zu sehen.
$T1$ und $T4$ stellen die Temperaturverläufe des dünnen und des
dicken Messingstabes dar. Es ist klar zu sehen das sich die
Temperatur beim dicken Messingstab $T1$ besser ausbreitet als
beim dünnen $T4$, da hier, wie zu erwarten, ein höherer Wärmestrom
auftreten kann. Für Aluminium $T5$ und Edelstahl $T8$ ist zu erkennen,
dass Aluminium die Wärme wesentlich besser leiten kann.
Insgesamt lässt sich sagen, dass die Wärmeleitfähigkeit
von dem Material abhängig ist und mit zunehmendem Durchmesser
steigt.
Nach 600s haben die Metalle folgende Temperaturen erreicht
\begin{align*}
  T_\text{Messing,dick} &= \SI{45,81}{\celsius}	& T_\text{Messing,dünn} &= \SI{44,08}{\celsius}\\
  T_\text{Aluminium} &= \SI{49,44}{\celsius}	& T_\text{Edelstahl} &= \SI{36,04}{\celsius}
\end{align*}

Wie oben zu erkennen ist, hat das Aluminium die höchste
Wärmeleitfähigkeit, der Edelstahl die niedrigste.

\begin{table}
  \centering
  \caption{Messwerte für $\increment T$.}
  \label{tab:dT}
  \sisetup{table-format=1.2}
  \begin{tabular}{S[table-format=3.0]SSSS}
    \toprule
    {$t$}&{$\increment T_\text{Messing,dick}$}&{$\increment T_\text{Messing,dünn}$}&{$\increment T_\text{Aluminium}$}&{$\increment T_\text{Edelstahl}$}\\
    \si{\second} & \si{\kelvin} & \si{\kelvin} & \si{\kelvin} & \si{\kelvin}\\
    \midrule
    100 & 4,53 & 4,94 & 2,98 & 8,16\\
    200 & 3,56 & 3,73 & 1,93 & 9,38\\
    300 & 3,07 & 3,28 & 1,62 & 9,27\\
    400 & 2,84 & 3,13 & 1,52 & 9,02\\
    500 & 2,73 & 3,05 & 1,47 & 8,88\\
    \bottomrule
  \end{tabular}
\end{table}

Aus \eqref{eqn:dQ} und den Werten aus \ref{tab:dT} ergeben sich
mit dem Abstand der Sensoren $\increment x= $ \SI{0,0303}{\meter}
die Wärmeströme

\begin{table}
  \centering
  \caption{Der Wärmestrom.}
  \label{tab:dQ}
  \sisetup{table-format=2.2}
  \begin{tabular}{S[table-format=3.0]SSSS}
    \toprule
    {$t$} & $\frac{\increment Q_\text{Messing,dick}}{\increment t}$ & $\frac{\increment Q_\text{Messing,dünn}}{\increment t}$
     & $\frac{\increment Q_\text{Aluminium}}{\increment t}$ & $\frac{\increment Q_\text{Edelstahl}}{\increment t}$\\
    \si{\second} & \si{\joule\per\second} & \si{\joule\per\second} & \si{\joule\per\second} & \si{\joule\per\second}\\
    \midrule
    100 & -0,86 & -0,55 & -1,11 & -0,26\\
    200 & -0,68 & -0,41 & -0,72 & -0,30\\
    300 & -0,58 & -0,36 & -0,60 & -0,29\\
    400 & -0,54 & -0,35 & -0,57 & -0,29\\
    500 & -0,52 & -0,34 & -0,55 & -0,28\\
    \bottomrule
  \end{tabular}
\end{table}

\begin{figure}
  \includegraphics{build/statdif.pdf}
  \caption{Die Temperaturdifferenzen $T2-T1$ und $T7-T8$}
  \label{fig:plotdif}
\end{figure}

Beim Vergleich der Temperaturdifferenzen an den Messstellen
von Messing und Edelstahl in \ref{fig:plotdif}, zeigt sich zunächst,
dass die bei Messing auftretenden Differenzen deutlich geringer
sind als die bei Edelstahl. Die Grundform beider Grafen ist dieselbe.
In den ersten Sekunden wächst die Temperatudifferenz an, da zunächst
nur die beheitzte Seite der Metallstäbe aufgewärmt wird, dann fällt sie
wieder ab sobald der Wärmestrom die unbeheitzte Messstelle erreicht, wobei
sie gegen einen festen Wert läuft. Dieser liegt bei Messing bei ca.
\SI{3}{\kelvin} und bei Edelstahl bei ca. \SI{8}{\kelvin}. Eine weitere
Auffälligkeit ist, dass der Graph von Messing eher und stärker
abfällt als der von Edelstahl und somit eine stärker definierte Form hat.
All diese Unteschiede lassen sich auf die höhere Wärmeleitfähigkeit
von Messing zurückführen, durch welche die Wärme die 2. Messstelle
schneller und effektiver erreicht.

\subsection{Auswetrung der dynamischen Messung}
\label{sec:adyn}

\begin{table}
  \centering
  \caption{Messgrößen für Messing.}
  \label{tab:dQ}
  \sisetup{table-format=3.2}
  \begin{tabular}{S[table-format=3.0]S[table-format=3.0]SSSS}
    \toprule
    $t_\text{Maxima nah}$ & $t_\text{Maxima fern}$ & $T_\text{Maxima nah}$
    & $T_\text{Minima nah}$ & $T_\text{Maxima fern}$ & $T_\text{Minima fern}$ \\
    \si{\second} & \si{\second} & \si{\kelvin}& \si{\kelvin}& \si{\kelvin}& \si{\kelvin}\\
    \midrule
     52 &  66 & 312.83 & 304,26 & 307,19 & 305,42 \\
    136 & 152 & 316,36 & 308,73 & 310,01 & 308,87 \\
    216 & 232 & 320,69 & 313,05 & 313,82 & 312,72 \\
    296 & 312 & 324,93 & 316,53 & 317,70 & 316,22 \\
    382 & 398 & 328,32 & 320,53 & 320,93 & 319,71 \\
    462 & 476 & 331,94 & 323,77 & 324,30 & 322,85 \\
    544 & 558 & 335,09 & 326,55 & 327,36 & 325,68 \\
    628 & 640 & 337,60 & 329,02 & 329,88 & 328,09 \\
    708 & 722 & 339,66 & 330,90 & 331,97 & 330,01 \\
    790 & 802 & 341,43 & 332,55 & 333,74 & 331,65 \\
    872 & 884 & 343,13 & 000,00 & 335,38 & 000,00 \\
    \bottomrule
  \end{tabular}
\end{table}

\begin{table}
  \centering
  \caption{Messgrößen für Aluminium.}
  \label{tab:dQ}
  \sisetup{table-format=3.2}
  \begin{tabular}{S[table-format=3.0]S[table-format=3.0]SSSS}
    \toprule
    $t_\text{Maxima nah}$ & $t_\text{Maxima fern}$ & $T_\text{Maxima nah}$
    & $T_\text{Minima nah}$ & $T_\text{Maxima fern}$ & $T_\text{Minima fern}$ \\
    \si{\second} & \si{\second} & \si{\kelvin}& \si{\kelvin}& \si{\kelvin}& \si{\kelvin}\\
    \midrule
      50 &  58 & 316,50 & 303,95 & 311,84 & 306,11 \\
     134 & 142 & 320,98 & 309,27 & 315,58 & 310,96 \\
     214 & 222 & 325,93 & 314,19 & 320,30 & 315,56 \\
     294 & 302 & 330,50 & 317,69 & 324,71 & 318,99 \\
     380 & 388 & 333,75 & 321,87 & 327,83 & 322,97 \\
     460 & 468 & 337,45 & 325,02 & 331,48 & 326,17 \\
     542 & 550 & 340,47 & 327,73 & 334,59 & 328,88 \\
     626 & 632 & 342,80 & 330,05 & 336,92 & 331,18 \\
     706 & 714 & 344,68 & 331,74 & 338,78 & 332,88 \\
     788 & 794 & 346,31 & 333,26 & 340,38 & 334,38 \\
     870 & 878 & 347,84 & 000,00 & 314,95 & 000,00 \\
    \bottomrule
  \end{tabular}
\end{table}

\begin{figure}
  \includegraphics{build/dyn80.pdf}
  \caption{Dynamische Messungen für Messing und Aluminium.}
  \label{fig:dynma}
\end{figure}

Zunächst werden die Amplituden des Schwingteils der Kurven
berechnet. Hierzu werden die Werte zweier nebeneinander liegender
Maxima gemittelt und der Wert des dazwischen liegenden Maximums wird
abgezogen. Dies wird für alle Maxima der jeweiligen Graphen getan.
Die Amplituden je eines Graphen, sowie die Phasendifferenzen, werden anschließend nach
\eqref{eqn:mean} gemittelt und nach \eqref{eqn:std} mit einem
Fehler versehen. Es ergeben sich die folgenden Werte

\begin{align*}
  A_\text{nah, Messing} &= \SI{9,81(22)}{\kelvin} & A_\text{fern, Messing} &= \SI{2,98(11)}{\kelvin}\\
  A_\text{nah, Aluminium} &= \SI{14,03(33)}{\kelvin} & A_\text{fern, Aluminium} &= \SI{6,94(28)}{\kelvin}\\
  \increment t_\text{Messing} &= \SI{14,18(159)}{\second} & \increment t_\text{Aluminium} &= \SI{7,64(77)}{\second}
\end{align*}

aus \eqref{eqn:Hz} und \eqref{eqn:lambda} ergeben sich

\begin{align*}
  f_\text{Messing} &= \SI{0,0122(3)}{\hertz} & \lambda_\text{Messing} &= \SI{0,1752(200)}{\meter}\\
  f_\text{Aluminium} &= \SI{0,0122(3)}{\hertz} & \lambda_\text{Aluminium} &= \SI{0,3254(338)}{\meter}
\end{align*}
Mit den Fehlern über \eqref{eqn:gauss}, \eqref{eqn:gaußf} und \eqref{eqn:gaußl}
\begin{align}
  \sigma_\symup{f}&=\left|\frac{-1}{T^2}\sigma_\symup{T}\right|
  \label{eqn:gaußf}\\
  \sigma_\symup{\lambda}&=\sqrt{\left(\frac{\increment x}{\increment t}\sigma_\symup{T}\right)^{\!\!2}
    +\left(\frac{-\increment x}{\left(\increment t\right)^{\!2}}T\sigma_\text{\increment t}\right)^{\!\!2}}
  \label{eqn:gaußl}
\end{align}
\begin{equation*}
    \sigma_\symup{\kappa} = \sqrt{
    \left(\frac{-\rho c\left(\increment x\right)^{\!2}}{2\left(\increment t\right)^{\!2}\ln{\frac{A_\text{nah}}{A_\text{fern}}}}\sigma_\text{\increment t}\right)^{\!\!2}
    +\left(\frac{-\rho c\left(\increment x\right)^{\!2}}{2 \increment t A_\text{nah}\left(\left(\ln{A_\text{nah}}\right)^{\!2}-\ln{A_\text{fern}}\right)}\sigma_\text{A,nah}\right)^{\!\!2}}\\
\end{equation*}
\begin{equation}
  \overline{+\left(\frac{-\rho c\left(\increment x\right)^{\!2}}{2 \increment t A_\text{fern}\left(\ln{A_\text{nah}}-\left(\ln{A_\text{fern}}\right)^{\!2}\right)}\sigma_\text{A,fern}\right)^{\!\!2}}
  \label{gaußk}
\end{equation}
mit der gemittelten Periodendauer $T$. Im folgenden wird die Wärmeleitfähigkeit $\kappa$ über \eqref{eqn:kappa}
mit dem Fehler \eqref{gaußk} berechnet und es ergibt sich.

\begin{align*}
  \kappa_\text{Messing} &= \SI{89,04(1044)}{\watt\per\meter\per\kelvin}\\
  \kappa_\text{Aluminium} &= \SI{198,47(2400)}{\watt\per\meter\per\kelvin}
\end{align*}

\begin{table}
  \centering
  \caption{Messgrößen für Edelstahl.}
  \label{tab:dQ}
  \sisetup{table-format=3.2}
  \begin{tabular}{S[table-format=4.0]S[table-format=4.0]SSSS}
    \toprule
    $t_\text{Maxima nah}$ & $t_\text{Maxima fern}$ & $T_\text{Maxima nah}$
    & $T_\text{Minima nah}$ & $T_\text{Maxima fern}$ & $T_\text{Minima fern}$ \\
    \si{\second} & \si{\second} & \si{\kelvin}& \si{\kelvin}& \si{\kelvin}& \si{\kelvin}\\
    \midrule
    116  &  180 & 316,76 & 304,86 & 305,05 & 304,43 \\
    318  &  396 & 324,75 & 312,49 & 309,01 & 308,70 \\
    520  &  596 & 331,19 & 318,18 & 313,36 & 312,98 \\
    720  &  788 & 335,76 & 322,25 & 317,11 & 316,53 \\
    918  &  986 & 339,07 & 325,16 & 320,13 & 319,28 \\
    1118 & 1178 & 341,63 & 327,02 & 322,56 & 312,36 \\
    1324 & 1384 & 343,42 & 000,00 & 324,43 & 000,00 \\
    \bottomrule
  \end{tabular}
\end{table}

\begin{figure}
  \includegraphics{build/dyn200.pdf}
  \caption{Dynamische Messung für Edelstahl.}
  \label{fig:dyne}
\end{figure}
